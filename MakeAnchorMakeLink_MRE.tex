\documentclass[12pt]{article}
\usepackage[a4paper, margin=1.5cm]{geometry}

\usepackage{hyperref} % For links and references
\usepackage{imakeidx}  % for the Indexing.


\usepackage{lipsum}   % For filler text

\usepackage{makeidx}  % for the index
\makeindex

% Helper to remove spaces (for MakeAnchor)
\makeatletter
\def\stripspaces#1{\zap@space#1 \@empty}
\makeatother

% Common function to normalize a name to a label-safe format
\makeatletter
\newcommand{\MakeLabelName}[1]{%
  \edef\tempa{\detokenize{#1}}%
  \expandafter\stripspaces\tempa%
}
\makeatother

% MakeAnchor: create a label with normalized name
\newcommand{\MakeAnchor}[1]{%
  \edef\tempa{\detokenize{#1}}%
  \edef\anchorname{\expandafter\stripspaces\tempa}%
  \label{\anchorname}%
  \index{#1}#1%
}

% MakeSkillAnchor: create a label with normalized name
\newcommand{\MakeSkillAnchor}[1]{%
  \edef\tempa{\detokenize{#1}}%
  \edef\anchorname{\expandafter\stripspaces\tempa}%
  \label{\anchorname}%
  \index{Skill!#1}#1%
}

% MakePerkAnchor: create a label with normalized name
\newcommand{\MakePerkAnchor}[1]{%
  \edef\anchorname{\MakeLabelName{#1}}%
  \label{\anchorname}%
  \index{Perk!#1}#1%
}

% MakeLink: display text, link to page of the anchor
\newcommand{\MakeLink}[1]{%
  #1 (P.\pageref{\expandafter\stripspaces\detokenize{#1}})\index{#1}%
}

% MakeSkillLink: display text, link to page of the anchor
\newcommand{\MakeSkillLink}[1]{%
  #1 (P.\pageref{\expandafter\stripspaces\detokenize{#1}})\index{Skill!#1}%
}

% MakePerkLink: display text, link to page of the anchor
\newcommand{\MakePerkLink}[1]{%
  \edef\anchorname{\MakeLabelName{#1}}%
  #1 (P.\pageref{\anchorname})\index{Perk!#1}%
}



% an example of its use inside a newcommand
\newcommand{\MakeBoldAnchor}[1]{%
  \textbf{#1}\MakeAnchor{#1}%
}

\begin{document}

\section{Introduction}

If you get ?? Try "Making it twice"

Here we will define some anchors and later link to them.

\subsection{First Concept}

We introduce the first concept called \textbf{My Concept One}.\MakeAnchor{My Concept One}

\lipsum[1]

Later we talk about the Skills \MakeSkillAnchor{Acrobatics} and \MakeSkillAnchor{Melee Combat}

\newpage

\subsection{Second Concept}

Another important idea is \textbf{Second Idea}.\MakeAnchor{Second Idea}

\lipsum[2]
\newpage

\subsection{Third Concept}

Introducing a third idea called \MakeBoldAnchor{Third Concept}.

\lipsum[3]
\newpage

\section{Referencing Concepts}

We can link to the concepts we defined earlier:

- Check \MakeLink{My Concept One} for details.

- See also \MakeLink{Second Idea}.

- Oh Yea and don't forget the \MakeLink{Third Concept}.

- You might also be interested in \MakeSkillLink{Melee Combat}.

\lipsum[3]
\newpage

\printindex

\end{document}
